\sectioncentered*{Введение}
\addcontentsline{toc}{section}{Введение}
\label{sec_introduce}
С момента создания глобальной сети интернет появилась необходимость в поиске и систематизации поступающей в сеть информации. Одним из возможных решений этой задачи являются сайты, предоставляющие доступ к определенным категориям данных – файловые хостинги. Пользователи этих сервисов могут искать необходимые данные среди уже загруженных на сервер, а также загружать свои собственные. Очевидно, что эффективность подобного рода сервисов довольно высока, так как пользователю не приходится тратить время на поиск информации среди различных источников. 

Появление файлового хостинга с базой актуальных данных обуславливает приток большого количества пользователей, что, в свою очередь, служит необходимым фактором для пополнения базы новой информацией, а также актуализации устаревших данных. Таким образом, создаётся устойчивая саморегулируемая система.

Ключевыми компонентами файлового хостинга являются база данных, поисковая система и система учёта пользователей. В совокупности эти компоненты максимально упрощаю задачу поиска информации для каждого пользователя сервиса.

Представленная дипломная работа посвящена задаче систематизации больших объемов данных. Решение этой задачи позволит существенно сократить время поиска информации, а также предоставит возможность обмена данными между пользователями.

Целью настоящей работы является эффективная реализация файлового хостинга с целью упрощения поиска необходимой пользователям информации, реализация возможности загрузки файлов на сервер хостинга, организация файлов по категориям, предоставление пользователям возможности конфигурировать параметры доступа к загруженным файлам.

Решаемые задачи:
\begin{itemize}
  \item реализация сервиса для систематизации информации;
  \item анализ условий использования файловых хостингов пользователями;
  \item анализ имеющихся файловых хостингов и их возможностей;
  \item понижение уровня сложности доступа к данным;
  \item автоматизация систематизирующих алгоритмов;
  \item удобство обмена файлами между пользователями сервиса;
  \item повышение надежности хранения информации;
  \item разработка структуры и реализация файлового хостинга с использованием перспективных технологий обработки данных.
\end{itemize}

В дипломном проекте описаны существующие решения указанной задачи, проведен их сравнительный анализ с проектируемой системой. Приведено описание разработанной автоматизированной системы. Кроме того, дипломный проект включает в себя руководство пользователя и руководство программиста (для корректного сопровождения программного продукта), а также главы, в которых произведена оценка качества и экономическое обоснование эффективности применения системы, рассмотрены вопросы безопасности и экологичности.
